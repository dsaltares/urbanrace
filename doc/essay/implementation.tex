% Implementation

In this section we will profusely explain some of the main features
from an implementation point of view. Each section will cover a specific
topic including a description of the problem, a explanation of the solution,
fragments of C\# code and diagrams when appropriate.

\section{Settings system}

In \game\ we have a lot of tunning parameters to configure different element's
behavior such as car physics properties, camera movement attributes, GUI
elements' position and so on. Hardcoding these values can result in a drastic
increase of development time, because every time we change one we have to
rebuild the module and their dependents.\\

One simple but effective solution could be having a settings file with
name value pairs. That way, we can change a value in that file so the next time
we run our application, the changes will be automatically applied. In our
game we use the \texttt{settings.xml} file which is read by the \textit{Settings}
singleton class as explained in the design section (page~\pageref{sec:design-system}).
Our settings file has the following schema.\\

\lstinputlisting[style=xml]{code/settings.xml}

To read the XML file we use the \textit{loadSettings()} static method in our
\textit{Settings} class. It basically uses the \textit{XDocument} library
in the C\# language to process the XML file. The values are stored in a dictionary,
a very appropiate data structure since it can use strings to index the elements
and element search only takes logarithmic time. Note that looking for an element
in a regular array or list involves linear time.\\

\lstinputlisting[style=CSharp]{code/loadSettings.cs}

To retrieve a value from our settings system we use the \textit{getOpt()} static
method in the \textit{Settings} class. This method uses the \textit{TryGetValue()}
method in the \textit{Dictionary} class to check element existence and perform
element retrieval at the same time. If we cannot find the desired option
we return \texttt{0.0f}.\\

\lstinputlisting[style=CSharp]{code/getOpt.cs}

\section{States management and game loop}

In \game\ we need to manage several states to represent different screens and transitions
between them as we mentioned during the design phase. Every state could potentially take
lots of memory so we cannot afford to load every state in main memory at the same time.
Some sort of state life-cycle management is in order. Our approach is shown in figure~\ref{fig:states-life}.

\image{states-life.png}{scale=0.5}{Urban Race states life cycle}{fig:states-life}{H}

Our \textit{UrbanRace} class has a \textit{Dictionary} containing states references
indexed by state types. At game initialization, this main class creates all the states in
the game but they remain unloaded. Then, we store in out \textit{currentState} attribute,
the reference to the main menu state and we load the mentioned state. When the \textit{changeState()}
method is called from somewhere in our code, we have to unload the current state, change the
current state value and then load the new current state. This is shown in the bit of code below.\\

\lstinputlisting[style=CSharp]{code/changeState.cs}

The current implementation has an important drawback which is worth mentioning. If a state
calls the \textit{changeState()} during its \textit{update()} process, it cannot make use
of resources after that call, because they will be unloaded after the state change operation.
A possible solution would have been to make the \textit{UrbanRace} class keep track of
state changes petitions so they are safely applied at the end of a game loop iteration. However,
the approach taken is enough for our game's scope.\\

During the \textit{Update()} and \textit{Draw()} phases, the correspondent method's in the
current state is called. During the update process, the input system is
updated along the audio engine. Furthermore, in XNA we have to prepare the system to render 2D
and 3D elements at the same time. In this case, our states render their 3D elements first, 
and then their 2D objects (graphical user interface). This means that before every draw
operation, we need to clean our deph buffer and rendering states so the 3D scene can be
properly rendered.

\lstinputlisting[style=CSharp]{code/gameloop.cs}

\section{Car and camera movement}
\label{sec:imp-car}

Implementing complex and realistic physics for our car movement was completely out of
this project's scope. That is why we adopted a quite simple approach. We can apply
a thrust or brake force to our car resulting in an acceleration. We can also steer
the wheels to change direction. To model this behavior we use linear cinematic properties
such as velocity, maximum speed, maximum acceleration and brake force. To represent
direction changes we have an angle (rotation around the vertical Z axis), angular
speed, angular acceleration and maximum angular acceleration. This approach is illustrated
in figure~\ref{fig:car-movement}.\\

\image{car-movement.png}{scale=0.4}{Car and camera model}{fig:car-movement}{H}

When we call the \textit{update()} method within the \textit{Car} class, we check
user input and calculate linear and angular forces. Then, these forces are applied
to obtain accelerations which are clamped so they cannot surpass the maximum. Then,
the angular and linear velocity are updated using the elapsed time and the acceleration.
Finally, the position and angle are also updated using velocities and the elapsed time
between frames. Note that we are using really basic dynamic physics and euler
integration. The \textit{update()} method is shown below.\\

\lstinputlisting[style=CSharp]{code/carUpdate.cs}

Our camera basically follows the car with a little delay when it steers so we can
see its sides. The \textit{Camera} class has similar physics properties like \textit{Car}.
However, it adds \texttt{distance} so we can position the camera behind de car, \texttt{height}
and a \texttt{aheadDistance} to look ahead of the car. This last property allows the player
to see what is in front of the car so he or she can avoid obstacles.\\

Every frame we update the camera position trying to make it stay behind the car
but limiting its acceleration. The code is shown below.\\

\lstinputlisting[style=CSharp]{code/updateCamera.cs}

\section{Event detection and response}

XNA's default input system is pretty easy to use but it lacks a proper event
system. For instance, it does not provide event handlers.
To achieve this feature we need to keep track
of the old and new states of the devices we want to support. To detect a button pressed
we need to check that in the previous state the button was released and in the current
state the same button is pressed. In our case, we want to develop a module to
detect keyboard, mouse and gamepad events so we can create cross-platform games
more easily.\\

Once per game loop iteration it is necessary to call the static \textit{update()}
method within the \textit{Input} class. This process involves getting the next state of
each device and comparing the new one against the old one to detect events. If a
callback for that event is registered, we need to fire it up so listeners can be notified.
The following code shows the whole process.\\

\lstinputlisting[style=CSharp]{code/inputUpdate.cs}

Our callbacks are simple C\# delegates as shown below. Depending on the type of event,
different types of data are passed as parameters to the event handler. Note that all
this follows the Observer design pattern as mentioned in the design section \cite{gamm94}.\\

\lstinputlisting[style=CSharp]{code/inputCallbacks.cs}

\section{Blender level creation and loading}

Keeping our level or track data inside our code would have been a really bad design decision.
That means that every time we want to make a small change to our level, like changing
a building position, we would need to rebuild the whole module (and its dependent modules).
Not to mention that the person responsible for designing levels would need to know how to code.
Furthermore, the level creation process would be really complicated and unorthodox. The solution
involves using a level designing tool and following a workflow similar to the one below.

\begin{enumerate}
	\item Create level in design tool.
	\item Export the level to a format the engine can process.
	\item Load the level at run-time.
\end{enumerate}

In \game\ we have used Blender \cite{hess07}, the open source 3D modelling and animation
tool as our level designer. Every object in the game has been modelled using this tool
and exported into the XNA compatible \texttt{.fbx} format. Later, the pieces are put together
to form a complete level. In the two following sections we will discuss the last two steps
in the workflow we detailed earlier because the first one (Blender usage) is out of this
report's scope.

\subsection{Level creation}

Once we have modelled our game elements, we need to make a composition to
create a whole level. When we are finished we will export the level into, what
is called, \textit{Dotscene} format. It basically is a simple XML file detailing
the name, position, orientation and scale of every object in our 3D scene. This
file can be easily processed by our \textit{Level} class.\\

The only thing we need is to adopt a naming convention so we can differentiate
simple geometry from collidable objects such as time bonuses, checkpoints or walls.
Our convention is fully detailed in the list following bullet list.

\begin{itemize}
	\item \textbf{Car}: there will always be a 3D car representing the initial position
	and orientation of our vehicle and it will be called \texttt{car}.
	\item \textbf{Time bonuses}: the objects that give the player extra seconds to finish
	the track will follow the pattern \texttt{time.seconds.number}. The number part is used
	to avoid name duplicities.
	\item \textbf{Checkpoints}: these objects will meet the pattern \texttt{checkpoint.number}
	where \texttt{number} means the order in which the checkpoint will need to be reached.
	\item \textbf{Collidable objects}: we can lay collidable objects in the scene with
	the pattern \texttt{scene.name.number}. \texttt{name} indicates the model that will
	be loaded. This object will be added to the \textit{CollisionManager}, so
	collisions tests against it will be performed. Then, it is recommended
	to add an entry to our \texttt{shapes.xml} file describing its collidable model.
	This file's schema will be fully detailed in the collision detection section (\ref{sec:collision-detection}).
	\item \textbf{Geometry}: we can add arbitrary geometry which will be passable by our car.
	This elements will need to follow the pattern \texttt{geometry.name.number}.
	\item \textbf{Skybox}: landscapes can be designed using the \texttt{skybox} keyword.
	\item \textbf{Terrain}: the terrain can be modelled and laid using the naming convention
	\texttt{Terrain.name}. How the terrain has to be exported will be detailed in 
	section~\ref{sec:terrain}.
\end{itemize}

Note that every model referenced in a level file will need to be registered in the
XNA content pipeline. Otherwise, the game will crash at run time.

\image{level-blender.png}{scale=0.25}{Level creation using Blender}{fig:level-blender}{H}

\subsection{Level loading}

Blender can export into the \textit{Dotscene} format using a plugin that can be found here.\\

\url{http://www.ogre3d.org/tikiwiki/Blender+dotScene+Exporter}\\

The XML files produced follows the schema shown below.\\

\lstinputlisting[style=xml]{code/level1.xml}

Our \textit{Level} \textit{load()} method uses the \textit{XDocument} library to
open and parse XML track files. It iterates through every node element
reading its position, orientation and scale, detects the element type
and calls the correspondent \textit{GameState} method to add the element. The
code is shown below these paragraph.\\

\lstinputlisting[style=CSharp]{code/levelLoad.cs}

\section{Collision detection}
\label{sec:collision-detection}

Collision detection is a critical subsystem in our game. We need to perform a big number
of tests per game cycle and we need a fairly precise collision detection to provide
a decent gameplay feeling. These tests may be resource intensive, so we need to be extra
careful in this matter. Many of the algorithms and approaches have been taken from the
Real-Time Collision Detection book \cite{eric05} although the techniques are used
in a wide collection of open source physics libraries.\\

In this section we will cover how our game objects' shapes are described, how we
have achieved run time collision test picking and how we perform collision detection in
global space.

\subsection{Shapes description}

We have talked that our game object may have collidable shapes that can be retrieved from
a collection our \textit{CollisionManager} keeps. These shapes are defined in an XML
file making the system quite customizable and extensible.\\

\lstinputlisting[style=xml]{code/shapes.xml}

The shape catalogue is initialized when the \textit{initShapeCatalog()} static method is
called. It basically iterates through every shape node in the xml file, it identifies
which type of shapes we are working with, and then it retrieves its properties. Note
that shapes are identified by the object name. Shapes are stored in a dictionary
indexed by their names.\\

\lstinputlisting[style=CSharp]{code/initShapeCatalog.cs}

When we want to retrieve a shape, the \textit{getShape()} method is called. It simply
looks for the shape name in the dictionary. If it finds it, it returns a copy of the
shape, otherwise it returns \texttt{null}.\\

\lstinputlisting[style=CSharp]{code/getShape.cs}

\subsection{Collision dispatching}

As we have already said, our collision detection system takes two shapes of the base class
and automatically determines which collision test needs to be called. This is achieved by
keeping a dictionary of collision tests method references (delegates in C\#) indexed by
a pair of simple integers. Then, every \textit{Shape} descendant has to implement a
\textit{getType()} method that returns a unique integer so we can identify the right
type of a base reference pointing to a descendant class object. When we want to check if
two shapes are colliding, we call the \textit{getCollision()} static method in the
\textit{Shape} class. It takes two general shape objects and it looks up their types
in the dictionary, if it finds a test for that shape combination, it calls it, otherwise
it returns false. This results in a safe and effective method to achieve run time collision
dispatching.\\

\lstinputlisting[style=CSharp]{code/collisionDispatching.cs}


\subsection{Collision tests}

Throughout this section we will discuss the implementation of every collision test provided
in the \game\ code. Note that not every collision test is used, however, the system has been
designed to be as extensible as possible. Every collision test is accompanied by a diagram
and the correspondent code.\\

The \textit{Sphere-Sphere} collision test is the simplest one. We only have to check if
the distance between both centers is smaller than the sum of both radius. Knowing that we are
comparing two lengths, we can use the squared length to improve performance. The tests is
illustrated in figure~\ref{fig:sphere-sphere}.\\

\image{sphere-sphere-test.png}{scale=0.5}{Sphere - Sphere csollision test}{fig:sphere-sphere}{H}

The implementation of this test is shown in the following fragment of code.\\

\lstinputlisting[style=CSharp]{code/sphere-sphere.cs}

To check if a \textit{Sphere} is colliding with an \textit{Axis Aligned Bounding Box} we first check
if the center of the sphere is inside the box. If not, we try to get the closest point in the box's
borders to the center of the sphere. If the sphere radius is greater than the distance, the two shapes
are colliding (this is shown in figure~\ref{fig:sphere-aabb}. Note that we represent bounding boxes
using their minimum and maximum corner positions.\\

\image{sphere-aabb-test.png}{scale=0.5}{Sphere - Axis Aligned Bounding Box collision test}{fig:sphere-aabb}{H}

A simple implementation is shown below.\\

\lstinputlisting[style=CSharp]{code/sphere-aabb.cs}

For the \textit{Axis Aligned Bounding Boxes} test we use the axis separation theorem.
This theorem says that two AABBs collide if and only if every axis (X, Y and Z)
projection overlap. The diagram in figure~\ref{fig:aabb-aabb} shows a 2D collision test
but we would only have to add one more axis if we wanted to represent it in 3D.\\

\image{aabb-aabb-test.png}{scale=0.5}{AABB - AABB collision test}{fig:aabb-aabb}{H}

This simple collision test is implemented below.\\

\lstinputlisting[style=CSharp]{code/aabb-aabb.cs}

The last and most complicated test is between two \textit{Oriented Bounding Boxes}. We are adding
a rotation to our previous AABB. The algorithm basically transform one of the OBBs into the
local coordinates of the other one. Then, we follow the separating axis theorem.\\

\image{obb-obb-test.png}{scale=0.5}{Oriented Bounding Boxes collision test}{fig:obb-obb}{H}

This quite complex algorithm has been extracted from the Real-Time Collision Detection
book \cite{eric05}.\\

\lstinputlisting[style=CSharp]{code/obb-obb.cs}

\subsection{Broad, extensive passes and callbacks}

We cannot simply iterate trough every pair of collidable objects in our game scene looking
for possible collisions. This would be an extremely inefficient and pointlent process because
some objects may be too far away to be colliding. We have adopted a simple approach to solve
this problem: broad and extensive passes. The algorithm is run every game loop iteration
inside the \textit{CollisionManager} \textit{checkCollisions()} method.\\

In \game\, we want to react to certain collisions between game objects in a specific way.
Our \textit{CollisionManager} has a collection of callback methods (delegates) indexed
by types pairs. Each type of object has an identifier so we can tell the manager to call
a specific method when two concrete object of a pair of given types collide.
For example, we can tell the manager
to call the \textit{collisionCarScene()} method when the car collides against our scene geometry.
Furthermore, we can call different methods depending on the type of collision: just started,
on going collision or collision finished.\\

The collision checking algorithm is the following in pseudocode.

\begin{enumerate}
	\item For every object pair
	\begin{enumerate}
		\item If they are inside the collision radius (10 distance units for instance)
			\begin{enumerate}
				\item If they were not colliding and now they are colliding
				\begin{itemize}
					\item Put them into the colliding objects collection
					\item Call begin callback if it exists.
				\end{itemize}
				\item If they were colliding and now they are still colliding
				\begin{itemize}
					\item Call during callback if it exists.
				\end{itemize}
				\item If they were colliding and now they are not colliding
				\begin{itemize}
					\item Remove object pair from colliding object collection.
					\item Call end callback if it exists.
				\end{itemize}
			\end{enumerate}
	\end{enumerate}
\end{enumerate}

\subsection{Car against scene callback}

One of the most interesting collision callbacks in the game is the one that reacts to 
collisions between the car and the scene. We basically get the collision normal vector,
rotate it so it reflects the angle between the car and the scene element. Then, we change the
car direction so it goes along the reflected vector direction. The new speed depends on the
car previous speed. We also play a crashing sound but this will be fully explained in following
sections. The process is illustrated in figure~\ref{fig:car-collision}.

\image{car-collision.png}{scale=0.4}{Car collision response diagram}{fig:car-collision}{H}

The algorithm is shown below.\\

\lstinputlisting[style=CSharp]{code/collisionCarScene.cs}

\section{Sound effects}
 
\game\ has both: background music and sound effects. The first one has been obtained from
Jamendo, the Creative Commons \cite{website:cc} music platform while effects have been
downloaded from several free resource banks. Playing music files is pretty straight forward
but we also needed dynamic sound to simulate engine speed variations and crashes intensity.
To achieve those effects the Microsoft Cross platform Audio Creation Tool (XACT 3) has
been used.\\

First, we had to create a sound project for our game and add every sound asset to it as
shown in figure~\ref{fig:xact}. This would be enough if we only wanted static sound but we
still needed to design an interface to work with our engine and crashing effects from the
code.\\

\image{xact.png}{scale=0.3}{Microsoft Crosss platform Audio Creation Tool}{fig:xact}{H}

When the car crashes, we want to set the crashing effect volume depending on the speed
the car bumped into the obstacle. That is why we created the \texttt{crashVolume} variable.
Inside our car-obstacle collision callback we only need to asign the volume a value ranging
from \texttt{0.0} to \texttt{1.0} as shown in the code below this paragraph.

\lstinputlisting[style=CSharp]{code/crashVolume.cs}

The engine makes two different sound effects, when the car is idle we want a vibrating engine
sound but when it is moving, we want it to make a racing car sound. Furthermore, we want
the intensity of the moving sound change depending on the car speed, we want its pitch to change.
That is why we created a Cue made of two different sounds. Using the \texttt{idleVolume} and
\texttt{engineVolume} variables we can turn the right one on and the other one off in a given
situation. Through the \texttt{idleVolume} variable we can tune the engine sound pitch effect.
This tasks is carried out inside the \textit{Car}'s \textit{update()} method.

\lstinputlisting[style=CSharp]{code/engineEffect.cs}

\section{Terrain}
\label{sec:terrain}

Our terrain is basically a squared grid with variable Z values designed in Blended at the
same time as the rest of the level. This mesh is exported to an XML format which is parsed
by our \textit{Terrain} class. We store the list of vertices and the list of indices that
form every triangle in the terrain. When we want to render the terrain we use the
\textit{DrawIndexedPrimitives()} passing the vertices and indices buffers as a triangle list.\\

The call to the terrain \textit{draw()} method is currently commented in the \textit{GameState}
class for performance reasons. The frame rate drops to 4 when the terrain has to be rendered.
Level of detail and occlusion algorithms need to be used to improve performance. The quadtree
approach was taken into account but could not be finished in time. However, terrain loading and
rendering still works (slowly).\\

In our level files, the terrain follows the pattern \texttt{Terrain.name} so we need to 
export the terrain with the right name.\\

\section{Skybox}

Our skybox is a \texttt{1x1x1} cube which normal vectors are looking inwards. The cube can also
be positioned, rotated and scaled to fit any scene. In fact, this positioning and scaling process
is performed during level design using Blender. We use a primitive similar to the one in the
terrain to draw the skybox. We have also used the graphics provided with the class examples.\\