% Design

Across this section we will discuss \game's classes design using UML diagrams \cite{larm02}
accompanied by brief explanations. For layout reasons we have divided the whole
system in several subsystems using their purpose as criteria to separate them.
For every design decision, a justification will be provided.\\

\section{System and states}
\label{sec:design-system}

Firstly, we have an application and states management layer which is responsible for
game's initialization, memory management between states and global logic such as
menu's navigation or game loop control. The classes involved in this subsystem are
shown in figure~\ref{fig:system-classes}.

\image{system-classes.png}{scale=0.5}{System classes design}{fig:system-classes}{H}

\begin{itemize}
	\item \textbf{UrbanRace}: main class which inherits from the XNA framework \textit{Game}
	class, it is the entrance point for our game. It basically initializes
	every other subsystem (such as collision detection, input handling or audio) and controls
	the game loop overriding its parent class methods. During the \textit{update()}
	phase it calls the active state \textit{update()} method and it does the correspondent
	process during the \textit{draw()} phase. It also handles state's changes trough its
	\textit{changeState()} public method.
	\item \textbf{Log}: static class following the Singleton design pattern \cite{gamm94}
	meaning that we will only have one single instance accessible from every other module
	in the system. This class is used to output information relevant to our game's execution
	so we can easily identify problems and bugs. It has three levels of logging: information,
	warning and error. The debug information is saved into a \texttt{log.html} file for
	easy and comfortable reading.
	\item \textbf{Input}: class also following the Singleton design pattern \cite{gamm94} which
	is responsible for capturing input events from the keyboard, mouse and gamepad. It is a
	highly reusable component because it is not game specific. It also provides a callback
	system following the Observer design pattern~\cite{gamm94}. This allows users to implement
	and register event handlers which will be called upon a specific type of event.
	\item \textbf{Camera}: the camera class is a tool that help us render our game's 3D scenes.
	It has been designed so it follows the car with a small delay so we can see the vehicle's
	sides. It has several tuning parameters such as \textit{distance}, \textit{height} or
	\textit{maxAngularAcceleration} to configure its behavior. More details will be provided
	in the implementation section (page~\pageref{sec:implementation}).
	\item \textbf{State}: this abstract class models a game screen (main menu, victory, defeat
	or game). It provides a mechanism to manage their memory usage through the \textit{load()}
	and \textit{unload()} methods. It also allows the system to update and render it through
	\textit{update()} and \textit{draw()}.
	\item \textbf{Settings}: this class also follows the Singleton design pattern \cite{gamm94}
	and provides a mechanism to configure all sorts of parameters in our game. It reads a
	\texttt{settings.xml} file containing name value pairs so we can change configuration
	settings without having to rebuild our code. This considerably speeds up our development
	process.
\end{itemize}

\section{Game state}

The Game state system is responsible for all the gameplay features: car control, collision detection
and response, level loading, camera following behavior, bonus picking and many more. The involved
classes are shown in figure~\ref{fig:gamestate-classes}.

\image{gamestate-classes.png}{scale=0.45}{Game state classes design}{fig:gamestate-classes}{H}

\begin{itemize}
	\item \textbf{GameState}: this class inherits from \textit{State} and implements the interface
	methods. It holds information regarding the current track such as its name, current lap, total
	laps, remaining time, total available time and the player's time. It also holds the \textit{Car}
	itself and other \textit{GameObjects} like the \textit{TimeBonuses}, the \textit{CheckPoints}
	or all the level geometry.\\
	
	The game state is also responsible for managing and rendering the \textit{Terrain} and the
	\textit{Skybox}, really important classes for environmental recreation purposes. The game state
	provides methods following the Observer pattern \cite{gamm94} to react to user interaction and
	collisions between game objects.\\
	
	This class delegates level loading and management to the \textit{Level} and \textit{TrackManager}
	classes respectively.
	
	\item \textbf{Level}: this class is responsible for loading and parsing levels stored in XML files and telling
	the \textit{GameState} which element to load and where (car, geometry, bonuses and checkpoints) through
	its \textit{load()} method. Levels are created using the Blender open source 3D modelling and animation
	tool~\cite{hess07} which process will be fully detailed in following sections.
	\item \textbf{TrackManager}: this manager also follows the Singleton design pattern \cite{gamm94}
	providing safe and global access to the list of available tracks in the game. The menu needs to know about
	the tracks so the player can select them, the game state needs to ask for a specific track to be loaded
	and the victory screen may need to register a new record. The track list and metadata is stored
	in an XML file called \texttt{tracks.xml}.
	
	\item \textbf{Skybox}: the \textit{Skybox} class is responsible for creating, managing and rendering the
	environment sky. This is noting more than a box with its normals looking inwards and six textures to imitate
	a regular landscape.
	
	\item \textbf{Terrain}: class responsible for terrain loading and rendering. Along the level, we can have
	height terrain created using Blender as well. It is basically a plane grid with different values in its
	Z (vertical) axis and a texture sticked to it.
\end{itemize}

\section{Game Objects}

This subsystem is responsible for modelling every object in \game: car, time bonuses, checkpoints and objects
in the scene like walls, buildings or the track itself. This subsystem's classes are shown in
figure~\ref{fig:gameobjects-classes}.

\image{gameobjects-classes.png}{scale=0.5}{Game objects classes design}{fig:gameobjects-classes}{H}

\begin{itemize}
	\item \textbf{GameObject}: this is our base class to represent object within our game. It encapsulates the
	object's graphical representation (\textit{Model}) with the collidable one (\textit{Shape}). It adds its location
	in our game world through the position (vector), orientation (quaternion) and scale (scalar) attributes.
	The \textit{GameObject} class provides an empty \textit{update()} method so every descendant can override it
	to customize its behavior. This class includes a \textit{draw()} method that simply renders the object model
	in the 3D scene applying the correspondent transforms (camera view and projection, translation, rotation and scale).
	Every game object has a current state which in our game can either be: normal or erase.\\
	
	The \textit{GameObject} class provides a static \textit{getCollision()} method that tests intersections
	between two objects' shapes. If they do not have shapes, it automatically returns false (one of them is 
	not a collidable object).
	
	\item \textbf{Car}: the \textit{Car} class is a \textit{GameObject} specialization and it adds custom behavior
	through the \textit{update()} method and several physics parameters to model its interaction with the world. We have
	adopted a extremely simplistic approach to this topic for scheduling and complexity reasons. Modelling and
	implementing realistic car physics without a physics simulation third party library can be highly tricky.
	Our car's physics model will be fully detailed in the implementation section (page~\pageref{sec:imp-car}).
	\item \textbf{TimeBonus}: the player has to collect time bonuses to increase his or her remaining time. These
	bonuses are modelled using this specialization of \textit{GameObject}. Time bonuses move up and down and spin
	around the Z axis so we have to add attributes to control this movement.
	\item \textbf{CheckPoint}: we have to make sure that the player drives across the whole track without cheating.
	The solution is setting up several invisible checkpoints that the player has to reach. The player has to go through
	every one of them in the proper order to successfully complete a lap. They only add two attributes: its order
	in the checkpoint sequence and the times the player has to go through them (laps).
\end{itemize}

\section{Collision detection}

The collision detection system provides a highly extensible mechanism
programmers can use to detect object intersections and react to them
using listeners through the Observer design pattern \cite{gamm94}. All
the classes involved are shown in figure~\ref{fig:collision-classes}.

\image{collision-classes.png}{scale=0.5}{Collision detection and response classes design}{fig:collision-classes}{H}

\begin{itemize}
	\item \textbf{Shape}: abstract class that models our objects' collidable shapes and provides a tests system to
	check collisions between two specific shapes. It has a transform to place the shape in world coordinates and
	a type to know the exact class of a given shape. This helps us to pick the right test between two shapes. We
	provide basic collision tests but the collection can be expanded through the \textit{addCollisionTest()} method. This
	class includes a static \textit{getCollision()} method to check collisions between two generic shapes,
	it automatically determines these shapes' type and picks the right test in a transparent way.
	
	\item \textbf{CollisionManager}: this manager keeps track of every active collidable object in our game. We can
	add or remove objects using the \textit{addObject()}, \textit{removeObject()} and \textit{removeAllObjects()} methods.
	Once per game loop iteration the \textit{checkCollisions()} method should be called to fire up collision callbacks
	upon intersections detection.\\
	
	The collision manager has a shape catalog which is read from an XML file and provides a description of every
	shape in the game. Every \textit{GameObject} can access this information to set up their collidable shapes
	at construction time using the \textit{getShape()} method. During game initialization we have to read that
	file through the \textit{initShapeCatalog()} method.\\
	
	We can register listener methods using the \textit{addCallback()} method. We pass a pair of game object types
	and a type of callback (begin, during and end). This implies that that callback will be called when two
	objects of the selected types start colliding, are colliding or stop colliding. We can also remove callbacks
	using the \textit{removeCallback()} or \textit{removeAllCallbacks()} methods.
	
	\item \textbf{Sphere}: this class models a simple sphere by adding a center position and a radius.
	
	\item \textbf{AxisAlignedBox}: we can model axis aligned boxes by adding their minimum corner position and
	their maximum corner position.
	
	\item \textbf{OrientedBox}: axis aligned bounding boxes do not provided very precise collision detection because
	they cannot be rotated in 3D space. By applying rotations we get oriented bounding boxes.
\end{itemize}