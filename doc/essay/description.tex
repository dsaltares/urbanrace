% Description

\section{About this document}

\game\ is a time attack racing game created as an individual project for the
Advanced Games Programming module at Kingston University. Throughout this
report we will profusely discuss every aspect of the game's development process.
The following bullet list summarizes the sections within this document.

\begin{enumerate}
	\item \textbf{Description}: in the current section we will explain the game
	mechanics and provide minimal and simple user instructions.
	\item \textbf{Design}: this block will cover the application development from
	a software design point of view. UML classes diagrams will be provided along classes
	description on a per component basis.
	\item \textbf{Implementation}: the main highlights and features implementation
	details will be explained throughout this section. Fragments of code will be
	attached when necessary.
	\item \textbf{Tools}: for further information purposes a list mentioning
	every used tool during any phase of the development process will be
	shown in this section.
	\item \textbf{Conclusion}: brief comment about the final and submitted product.
\end{enumerate}

\section{Game design}

\game\ is a time attack racing game for Windows although, potentially, it could be
ported to Xbox 360 and Windows Phone 7 with minimal changes because it has been
developed using the C\# programming language and XNA Game Studio 4.0.\\

The player drives a yellow car and has to get to the goal before the time expires.
Note that some tracks may demand more than one lap, in fact, they could ask for an
indeterminate number of laps. The tracks are designed so they cannot be finished
on time as they are. The player has to collect time bonuses, which are represented
by clocks and are spreeded across the track. These bonuses give the player extra seconds,
so he or she can finish on time.\\

The game is composed by several screens connected together as shown in
figure~\ref{fig:states}.\\

\image{states.png}{scale=0.5}{Game states flow}{fig:states}{H}

When \game\ is run, the track selection menu is shown and when the user chooses a track,
the game starts. If the player finishes the track on time, the victory screen is presented,
otherwise, the player is taken to the defeat menu. From either of those, the player
can retry the track or go back to the main menu to select a different track. If the player
beats his or her own record, it will be shown in a panel within the victory screen.\\

During the game, the player will drive the car with simple controls: steering, thrust
and brake. He or she basically will have to avoid crashing into the level walls to gain
some time. For simplicity reasons, an arcade approach has been taken, meaning that no
realistic physics has been implemented. The game is meant to be simple and fun.\\

\section{Game controls}

Inside the game menu, the player can browse different options using the \textit{arrow}
keys and he or she can select a specific one using the \textit{enter} key. During the
game, the controls are shown in figure~\ref{fig:controls}.

\image{controls.png}{scale=0.5}{Game controls during a race}{fig:controls}{H}
